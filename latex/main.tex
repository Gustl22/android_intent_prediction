% !TeX spellcheck = en-US
% !TeX encoding = utf8
% !TeX program = pdflatex
% !BIB program = biber
% -*- coding:utf-8 mod:LaTeX -*-


% vv  scroll down to line 200 for content  vv


\let\ifdeutsch\iffalse
\let\ifenglish\iftrue


\input{pre-documentclass}
\documentclass[
  %
  %ngerman, %%% Add if you write in German.
  %
  % fontsize=11pt is the standard
  a4paper,  % Standard format - only KOMAScript uses paper=a4 - https://tex.stackexchange.com/a/61044/9075
  twoside,  % we are optimizing for both screen and two-side printing. So the page numbers will jump, but the content is configured to stay in the middle (by using the geometry package)
  bibliography=totoc,
  %               idxtotoc,   %Index ins Inhaltsverzeichnis
  %               liststotoc, %List of X ins Inhaltsverzeichnis, mit liststotocnumbered werden die Abbildungsverzeichnisse nummeriert
  headsepline,
  cleardoublepage=empty,
  parskip=half,
  %               draft    % um zu sehen, wo noch nachgebessert werden muss - wichtig, da Bindungskorrektur mit drin
  draft=false
]{scrbook}
\input{config}


\usepackage[
  title={Intent Prediction with Vectorized Sequential Android UI Tree Data}, % Do not forget to capitalize your title correctly, you may use the following page to help you: https://capitalizemytitle.com/
  author={August Oberhauser},
  email={august.oberhauser@campus.lmu.de},
  type=master,
  institute={Institut für Informatik}, % or other institute names - or just a plain string using {Demo\\Demo...}
  course={Medieninformatik},
  examiner={Prof.\ Dr.\ Sven Mayer},
  supervisor={Florian\ Bemmann,\ M.Sc.},
  startdate={June 20, 2023},
  enddate={January 8, 2024},
  % Falls keine Lizenz gewünscht wird bitte auf "none" setzen
  % Die Lizenz erlaubt es zu nichtkommerziellen Zwecken die Arbeit zu
  % vervielfältigen und Kopien zu machen. Dabei muss aber immer der Autor
  % angegeben werden. Eine kommerzielle Verwertung ist für den Autor
  % weiter möglich.
  copyright=ccbysa, % ccbysa, ccbynosa, cc0, none
  language=english
]{lmu-thesis-cover}

\newacronym[description={Error rate}]{er}{ER}{error rate}
%\newacronym[description={Scientific approach to form statistical models without the need to explicitly program it}, plural={RDBMS},shortplural={RDBMS}]{rdbms}{RDBMS}{Relational Database Management System}
\newacronym[description={Scientific approach to form statistical models without the need to explicitly program it}]{ml}{ML}{Machine Learning}

\newglossaryentry{gl-bigdata}{name={Big Data}, description={Extremely large and complex data sets which can only be processed with modern computing soft- and hardware}}


\makeindex

\begin{document}

%tex4ht-Konvertierung verschönern
\iftex4ht
  % tell tex4ht to create picures also for formulas starting with '$'
  % WARNING: a tex4ht run now takes forever!
  \Configure{$}{\PicMath}{\EndPicMath}{}
  %$ % <- syntax highlighting fix for emacs
  \Css{body {text-align:justify;}}

  %conversion of .pdf to .png
  \Configure{graphics*}
  {pdf}
  {\Needs{"convert \csname Gin@base\endcsname.pdf
      \csname Gin@base\endcsname.png"}%
    \Picture[pict]{\csname Gin@base\endcsname.png}%
  }
\fi

%\VerbatimFootnotes %verbatim text in Fußnoten erlauben. Geht normalerweise nicht.

\input{commands}
\pagenumbering{arabic}
\Coverpage
% \Copyright
%Eigener Seitenstil fuer die Kurzfassung und das Inhaltsverzeichnis
\deftriplepagestyle{preamble}{}{}{}{}{}{\pagemark}
%Doku zu deftriplepagestyle: scrguide.pdf
\pagestyle{preamble}
\renewcommand*{\chapterpagestyle}{preamble}


\section*{Abstract}
\label{sec:abstract}

\todo{THIS IS THE EXPOSÉ, no need to correct!!}
\todo{search keywords to replace: you, we, I, TAKEAWAY, ??}
\todo{check if all images and tables are referenced}
\todo{check if words from the glossary can be replaced in the text}
\todo{check if every section or chapter cross reference has a prefix "chapter" or "section"}


The intent of a user is a source of information, which is hard to accommodate.
In the context of using a smartphone, this can imply the next user action, the user flow, or the recommendation for the next app.
A flexible, open-source proof-of-concept for an intent prediction model based on a \gls{lstm} has been worked out.
It includes three models ClickOnly, SelectedFeatures and FeaturesClickShifted, which aim to predict the next user click.
All three are performing only slightly better than the statistically mean click gesture, trained on 15\% of the usable Rico dataset.
Criteria for a dataset suitable for Android intent prediction were determined.
Approaches, like transformer multi-attention model, actionable element prediction, and further metrics have potential to push the development of semantically meaningful prediction models.


\cleardoublepage


% BEGIN: Verzeichnisse

\iftex4ht
\else
  \microtypesetup{protrusion=false}
\fi

%%%
% Literaturverzeichnis ins TOC mit aufnehmen, aber nur wenn nichts anderes mehr hilft!
% \addcontentsline{toc}{chapter}{Literaturverzeichnis}
%
% oder zB
%\addcontentsline{toc}{section}{Abkürzungsverzeichnis}
%
%%%

%Produce table of contents
%
%In case you have trouble with headings reaching into the page numbers, enable the following three lines.
%Hint by http://golatex.de/inhaltsverzeichnis-schreibt-ueber-rand-t3106.html
%
%\makeatletter
%\renewcommand{\@pnumwidth}{2em}
%\makeatother
%
\setcounter{secnumdepth}{4}
\setcounter{tocdepth}{4}
\tableofcontents

% Bei einem ungünstigen Seitenumbruch im Inhaltsverzeichnis, kann dieser mit
% \addtocontents{toc}{\protect\newpage}
% an der passenden Stelle im Fließtext erzwungen werden.

\listoffigures
\listoftables

% Control List of Listings
\let\iflistings\iffalse
%Wird nur bei Verwendung von der lstlisting-Umgebung mit dem "caption"-Parameter benoetigt
%\lstlistoflistings
%ansonsten:
\iflistings
  \ifdeutsch
    \listof{Listing}{Verzeichnis der Listings}
  \else
    \listof{Listing}{List of Listings}
  \fi
\fi

% Control List of Algorithms
\let\ifalgorithms\iffalse
\ifalgorithms
  %mittels \newfloat wurde die Algorithmus-Gleitumgebung definiert.
  %Mit folgendem Befehl werden alle floats dieses Typs ausgegeben
  \ifdeutsch
    \listof{Algorithmus}{Verzeichnis der Algorithmen}
  \else
    \listof{Algorithmus}{List of Algorithms}
  \fi
  %\listofalgorithms %Ist nur für Algorithmen, die mittels \begin{algorithm} umschlossen werden, nötig
\fi

% Control Glossary
\let\ifglossary\iftrue
\ifglossary
%  \setglossarysection{section} remove empty sites
%  \clearpage
  \printnoidxglossaries
%  \clearpage
\fi

\iftex4ht
\else
  %Optischen Randausgleich und Grauwertkorrektur wieder aktivieren
  \microtypesetup{protrusion=true}
\fi

% END: Verzeichnisse


% Headline and footline
\renewcommand*{\chapterpagestyle}{scrplain}
\pagestyle{scrheadings}
\pagestyle{scrheadings}
\ihead[]{}
\chead[]{}
\ohead[]{\headmark}
\cfoot[]{}
\ofoot[\usekomafont{pagenumber}\thepage]{\usekomafont{pagenumber}\thepage}
\ifoot[]{}


%% vv  scroll down for content  vv %%


%%%%%%%%%%%%%%%%%%%%%%%%%%%%%%%%%%%%%%%%%%%%%%%%%%%%%%%%%%%%%%%%%%%%%%%%%%%%%%
%
% Main content starts here
%
%%%%%%%%%%%%%%%%%%%%%%%%%%%%%%%%%%%%%%%%%%%%%%%%%%%%%%%%%%%%%%%%%%%%%%%%%%%%%%

\chapter{Introduction}
\label{sec:introduction}

The interaction of a user with an end device such as a smartphone or a computer is very diverse and difficult to contextualize.
Nevertheless, user-specific (personalized) as well as global (collaborative) patterns can possibly be worked out with the help of preceding user interactions and screen contents.
These could be used to predict the intention, such as the action or the motivation of a user or a group of users.
Especially the level of detail, at which these predictions can be made, is an integral part of offering a reliable service.
By making use of the continuous on-device data, the users behavior can be worked out and be used to forecast their next actions.
These can then be very coarse, such as predicting the next app.
Or they can be very detailed, e.g.\ determining the next user action, such as filling out a form field or selecting the favorite meal in the delivery service app.

\section{The Role of Intent Prediction}
\label{sec:role-intent-prediction}

The term \ti{intent} can have ambiguous meanings and can be used in different context.
In the dictionary it is described as the fact of intending something, so it is planned to do~\cite{dictionaryIntent}.
Generally we can assume that intent is a stronger desire to accomplish ones intension.
Kofler and Hanjalic et al.~\cite{kofler2016user} also describe the intent as \quotes{immediate reason, purpose, or goal [\ldots] that motivates a user to} act.

The aim or purpose must be differentiated from the actual user input, also known as interaction.
So the intent can be seen as the preliminary stage of interaction.
Gestures and click sequences then are the concrete actions, and might fulfill a (small) part of the users intent.
Therefore, in this work it is not the goal to obtain the users full intent, but to work out factors, such as user inputs or gestures, which hint to the intent of the user.
Further, \ti{prediction} describes that the intent or factors of intent should be available (calculated) before they have actually happened or were measured.

\todo{Explain screen, view, flow, click}

But it can be easily replaced by more semantic data, which might even be easier to predict.
Descriptive User intent embedding as preliminary stage, which can be applied with Screen2Words\cite{wang2021screen2words} in a similar fashion.
difference intent / interaction / screen / flow / click


\section{Necessity of Vectors for Android UI}
\label{sec:necessity-of-vectors-for-android-ui}
\todo{Explain how many time a user spents on a mobile device, facilitate steps, make it more productive, qualitative user experience}


Motivation for transforming Android UI tree data to vectors

- open source code for predicting next user click / action
- evaluate and compare existing approaches
- provide tools to reduce screen sequences to vectors
- how to work with multidimensional (multi-modal) data in RNNs

- Low Button depth: number of clicks until one gets to the action \cite{lee2018click}


% Zusammenfassen in die wichtigste Aussage: machine learning ist spannend für die Vorhersage weil es Usergruppen erstellt, die sensiblen Daten rausgefiltert werden und ....?
Furthermore, such a technology provides many benefits in addition to intent prediction.
The complexity and the size of a \gls{ui} tree can be reduced by vectorizing them. \todo{vectorization}
User groups can be worked out in social studies, which have a similar behavior when using digital \gls{ui} systems \cite{jayarajah2015need}.
Also, the technical expertise on individual features can be eliminated, that would be required to manually compare user sessions \cite{ghods2019activity2vec}.
A users smartphone usage history can be saved or compared, considering the aspect of avoiding saving personal information.
App developers can be supported to improve their app design and usability.
It may be applied in psychology and market research or help detecting addictions.
On a technical side the method can optimize preloading of processes on mobile devices, which results in energy savings \cite{shen2019deepapp}.
As shown, many fields of application can profit by elaborating such a system.
It would be exciting to know, how the concrete concept would look like and if it can be implemented successfully e.g. to improve the user experience on end-user devices.

For this purpose, the Android \gls{ui} of the device can be tracked, and relevant information are collected.
After a few preprocessing steps, the data can be trained with a \gls{ml} model to acquire the user behaviors and then make predictions, if convenient for the user.

\begin{figure}[htbp!]
    \centering
    \includesvg[width=\textwidth]{graphics/vectorization.svg}
    \caption[Schema of ML-algorithm predicting user intent]{
        Possible procedure using a Machine-Learning algorithm to predict the next intent from a beginning user session:
        The input (1) can be a sequence of Android tree data.
        With help of a Machine-Learning-Model (2) (e.g. RNN) a vector representation can be trained and then predict the most probable action or screen (3) from a given starting sequence, but also can be improvde through the users feedback.
    }
    \label{fig:encode-decode}
\end{figure}

A guideline and a proof-of concept will be developed on how a model for predicting user intent could be built.
To this end, possibilities for collecting and vectorizing sequential UI trees (e.g., from the Android Accessibility Service) will be discussed.
The performance of the model can be measured, for example, by indicators such as the amount of training data and time spent on the learning process.

\todo{formulate contributions}

Contributions: \cite{zhou2021large}
• An analysis of a large-scale dataset of mobile user click se-
quences that reveals rich factors and complexity in modeling
click behaviors, which contributes new knowledge to under-
stand mobile interaction behaviors.
• A Transformer-based deep model that predicts next element
to click based on the user click history and the current screen
and time. The model does not rely on a vocabulary of prede-
fined UI elements and provides a general solution for model-
ing arbitrary UI elements for click prediction.
• A thorough experiment that compares our deep model with
multiple alternative designs and baseline models, and an
analysis of model behaviors and benefits that the model can
bring to improve mobile interaction.

Contributions: \cite{li2021screen2vec}
Screen2Vec: a new self-supervised technique for generating
more comprehensive semantic embeddings of GUI screens
and components using their textual content, visual design
and layout patterns, and app meta-data.
(2) An open-sourced GUI embedding model trained using the
Screen2Vec technique on the Rico [9] dataset that can be
used off-the-shelf.
(3) Several sample downstream tasks that showcase the model’s
usefulness.

In computer science there often coexists multiple (correct) solutions to the same problem.
Many technologies have the capabilities of achieving a similar result, but are specialized in the one or the other way.
That led to the first question

\todo{formulate research questions}

\quotes{What is a suitable model for the prediction of user intent?}.

\quotes{At what level of detail the predictions can be made?}


%\label{subsec:motivation}
%\todo{P1.1. What is the large scope of the problem?}
%\todo{P1.2. What is the specific problem?}
%
%% Second Paragraph
%% CORE MESSAGE OF THIS PARAGRAPH:
%\todo{P2.1. The second paragraph should be about what have others been doing}
%\todo{P2.2. Why is the problem important? Why was this work carried out?}
%
%% Third Paragraph
%% CORE MESSAGE OF THIS PARAGRAPH:
%\todo{P3.1. What have you done?}
%\todo{P3.2. What is new about your work?}
%
%% Fourth paragraph
%% CORE MESSAGE OF THIS PARAGRAPH:
%\todo{P4.1. What did you find out? What are the concrete results?}
%\todo{P4.2. What are the implications? What does this mean for the bigger picture?}

%LaTeX hints are provided in \autoref{chap:latexhints}.


\chapter{Theoretical Framework}
\label{ch:theoretical-framework}

Before starting with rehabilitate the main topic, some basic concepts, technical terms and underlying theory has to be explained.
This is important to faster read through the thesis without the need to interrupt the flow of thought.
In the following, a set of established theories and terms is explained which are not directly related to the thesis formulation, but essential to comprehend numerous contexts.

\section{Android UI}
\label{sec:android-ui}

To be able to use any of the things displayed on the mobile device, some concepts of \gls{ui} programming must be shown.
A \gls{ui} enables the user to view the applications data on the screen but also to interact with the device especially on mobile devices \cite{android_ui_layer}.

The main challenge is to bring the application data in the right format, so that the display can interpret the instructions to draw the elements.
Each mobile phone with the Android \gls{os} has a basic set of native functions through which the \gls{ui} can be drawn and updated, a so-called \gls{api}.
These functions can be very general to instruct drawing a whole component such as an alert box, or they can be very specific as drawing single rectangles in a canvas.

The rough transition from the application data (data layer) to the display data (ui layer) is depicted in figure \ref{fig:android_udf}.
The application data is transformed, concatenated or filtered to be saved in a view model, which represents the state for each view.
The view model is then layed out to multiple \gls{ui} elements.
E.g.\ they are loaded in the Android activity via view layouts~\cite{android_draw_views} or composed in a declarative approach~\cite{android_jetpack_compose}.
It is generally advised to use a \gls{udf}.
This ensures that the data is only changed in one place and doesn't get out of sync between \gls{ui} elements, the view model and the data layer.
The \gls{ui} can also register user inputs (like a button press) and report them back to the view model.
The view model then updates the application data, if needed, and then also reports the current \gls{ui} state back to the UI elements to be rerendered.

\begin{figure}[htbp!]
    \centering
    \begin{subfigure}[b]{0.5\textwidth}
        \centering
        \includegraphics[width=\textwidth]{graphics/android_udf}
        \caption{Diagram of \gls{udf} in app architecture \cite{android_ui_layer}}
        \label{fig:android_udf}
    \end{subfigure}
    \hfill
    \begin{subfigure}[b]{0.4\textwidth}
        \centering
        \includegraphics[width=\textwidth]{graphics/android_semantics-ui-tree}
        \caption{Schema of how the semantics tree is related to the \gls{ui} hierarchy \cite{android_semantics_compose}}
        \label{fig:android_semantics_ui_tree}
    \end{subfigure}
    \caption{Structure of the Android \gls{ui}}
    \label{fig:android_tree}
\end{figure}
%https://developer.android.com/guide/topics/ui/how-android-draws \cite{android_draw_views}
%https://developer.android.com/jetpack/compose/mental-model \cite{android_jetpack_compose}

\subsection{Data tree structure}
\label{subsec:data-tree-structure}

The \gls{ui} elements (i.e.\ the composition) themselves are hierarchically structured in a tree.
This allows the renderer to calculate relative distances, floatings and skip processing hidden or overlapped elements.

With the Android Layout Inspector (figure~\ref{fig:android_layout_inspector}) a view hierarchy tree can be visually inspected while displaying its position and layout on the Android screen.
Also, the layout attributes can be validated.
This tool allows to debug complex \gls{ui}s especially when using nested components and display them in a simplistic way.
Note that this tool is only available if one has access to the app's source code.

\begin{figure}[htbp!]
    \centering
    \includegraphics[width=\textwidth]{graphics/android_layout_inspector}
    \caption{The Android Layout Inspector tool using the example of \quotes{App ins Grüne}~\cite{mimuc_app_ins_gruene} by the Media Informatics Group of the LMU in Munich.}
    \label{fig:android_layout_inspector}
\end{figure}

\subsection{Android Accessibility Service}
\label{subsec:android-accessibility-service}

If an app is running in production on a users device, meaning that the app is compiled and publicly available, the ways of accessing the Android \gls{ui} tree are limited.
This behavior is of course wanted for safety and privacy reasons.
Nonetheless, if desired, a user can explicitly allow certain apps to gain access to the semantics tree of your Android \gls{os}.
This is especially useful for providing accessibility services for impaired users (like done with the TalkBack app).
Or -- as in our case -- the setting can be used to enable services which collect data for user studies or scientific experiments.

This semantics tree is deviated from the existing \gls{ui} tree.
It can be fed via special semantic properties while composing the \gls{ui}, e.g.\ by specifying the \code{contentDescription} property of an icon \cite{android_semantics_compose}.
Providing semantics is not limited to native platforms as shown by Flutter~\cite{flutter_semantics} and React Native~\cite{react_native_accessibility}.
In figure~\ref{fig:android_semantics_ui_tree} a schema is presented, which shows how the elements of the semantics tree are spanned compared to the components on the \gls{ui} layer.

%https://github.com/android/codelab-android-accessibility
To take advantage of the semantics tree, a custom accessibility service can be built, which can run in the background.
This service tracks all UI changes and has access to the current view hierarchy of the screen, which also inherits the semantic tree.
By altering the code of the \ti{AccessibilityNodeInfoDumper} one can extract the view hierarchy to a locale or remote database~\cite{android_accessibility_node_info_dumper}.
In the code section \ref{android_accessibility_node} a small fraction of a view hierarchy is shown.
It contains nested nodes with various attributes which represent the components of the combined \gls{ui} and semantics hierarchy.

%Semantics tree:
%https://api.flutter.dev/flutter/widgets/Semantics-class.html
%https://developer.android.com/jetpack/compose/semantics
%https://android.googlesource.com/platform/frameworks/testing/+/jb-dev/uiautomator/library/src/com/android/uiautomator/core/AccessibilityNodeInfoDumper.java \cite{android_accessibility_node_info_dumper}
%https://github.com/Gustl22/android-accessibility/blob/c158808533d6fc017455184a7317555d3e6946f6/GlobalActionBarService/app/src/main/java/com/example/android/globalactionbarservice/uiautomator/AccessibilityNodeInfoDumper.java

%Mean 18 actionable elements, with Std=12. \cite{zhou2021large}

\lstinputlisting[language=XML,label=android_accessibility_node,caption={Android Accessibility Node in XML.},float]{code/android_accessibility_node.xml}

\section{Machine Learning}
\label{sec:machine-learning}

\gls{ml}, a term spread by Arthur Lee Samuel, is a method of data analysis, more precisely a scientific approach to form statistical models without the need to explicitly program it~\cite{mahesh2020machine}.
It uses algorithms to iteratively learn how data is structured.
In contrast to statistical inference or manually crafted statistical models respectively, \gls{ml} can solve tasks by automation of model building.
Its advantages lie in finding hidden relations and patterns from the context, without having any or only a small pre knowledge of the data, thus it is a strong tool for generalization or abstraction of large datasets, also known as \gls{gl-bigdata}.
\gls{ml} can be applied to the following fields among others: email and spam filtering, fraud detection, cybersecurity, web search engines, recommender systems (like known from Netflix or Amazon), advertising, translators and text generation, pattern and image recognition.
The data driven approach also comes with some drawbacks: the outcome heavily depends on the provided data.
It can include biases and therefore may acquire forms of discrimination or unfair treatment.
Nonetheless \gls{ml} has a lot of potential to uncover hidden connections in large datasets.
The most important concepts are explained to be able to follow certain decisions in this thesis.

\todo{explain Tensors, Datasets}
%https://stackoverflow.com/a/48599383/5164462

\subsection{Preprocessing}
\label{subsec:preprocessing}

Preprocessing describes the step after one acquired their data, but before training the \gls{ml} model.
This step is not to be underestimated.
A \gls{ml} model can perform significantly better when certain preprocessing steps are applied \cite{alam2019impact}.

To be able to preprocess, we have to know with what kind of data we handle with.
Data entries can occur in different forms, but we can break them down in three main types:
\begin{itemize}
    \item \tb{Categorical values}: a value is always assigned to a class with a fixed pool of predetermined classes.
          E.g.\ letters, words, brands, animals, chemical elements
    \item \tb{Continuous values}: the value can be fractional and may lies in between a lower and an upper bound.
          E.g.\ temperature, velocity, geographic position
    \item \tb{Integer values}: the value is a whole number and may also lie in between a lower and an upper bound.
          E.g.\ revolutions per minute, product number, annual sales
\end{itemize}

For all types -- discrete, continuous, and categorical values -- we have a wide variety of options for preparing them in order to be subsequently processed by a \gls{ml} model \cite{duong2021}.
These are listed in the following, while not all can be applied to all types of values.

\subsubsection{Feature selection}

Feature selection is a crucial step to successfully develop a model with the desired results.
It can improve learning performance, thus reduces time, increases computational efficiency, decreases memory storage, and helps build better generalization models \cite{li2017feature}.
Also, it may be a valid approach to get around missing data and can help structure the data by removing unnecessary clutter.

%https://dl-acm-org.emedien.ub.uni-muenchen.de/doi/pdf/10.1145/3136625
On the technical sight, feature selection can be differentiated for two goals: supervised and unsupervised learning~\cite{li2017feature}.
However, principles from the supervised feature selection can be applied in the unsupervised domain, resulting in a semi-supervised \ref{semi-supervised} filter selection.
For classifiers and regression problems (unsupervised) following methods can be applied.
Multiple features can be compared by calculating their correlation.
If one feature is uncorrelated to all other features, this may be an indicator that this feature can be dropped, as it doesn't contribute to the resulting model (\ti{filter method}).
However, this can only be stated for linear correlations, thus it can contribute to the result in an unpredicted way.
Also, if two features correlate too much to each other, one feature probably is redundant and can be dropped.
A good approach is also to reduce the dimensionality of the input data, e.g.\ by replacing~\gls{gl-one-hot} encoded features of the same domain with embeddings (\ti{embedded method}).
This is also called \ti{feature extraction}.
Feature extraction can also be used in unsupervised feature selection.
Clustering is a common approach to reduce the number of input dimensions by gaining insights of which classes can be merged and which need to stay.
A more computational but promising solution is to filter the features by optimizing the model result, also called a \ti{wrapper method}.
By gradually removing and adding features and calculating the models performance, one can determine which inputs are important and which are at risk to increase computing time without noticeable effect.

\todo{formulate or remove}
semantic or legal feature selection
Such as Filtering privacy invasive details
Remove sensitive data
%https://news.mit.edu/2023/new-way-look-data-privacy-0714

Parameterizing the vectorization process
a) Vector length
b) Weighting of features
c) Manipulating individual parameters of model

\subsubsection{Missing data}
Some data entries may are missing.
Therefore, you have two approaches to get around these missing values.
One can drop these values by removing the column or row.
This is only recommended if you are not relying on this data entry, or this the whole feature is not expected to be important enough to bring any value to the model's performance.
Further you can fill the data with a default value like zero or calculate a reasonable value from the surrounding data entries by taking their \quotes{mean, median, or interpolation}~\cite{duong2021}.

\subsubsection{Normalization and Standardization}

Many \gls{ml} models work better or exclusively with normalized data.
This means that the values have to be in a certain range, most commonly are from \tb{0} to \tb{1} or from \tb{-1} to \tb{1}.
This can be achieved by dividing all values with the difference of the minimum and maximum value and shift the output accordingly \cite{duong2021}.

% X new = (X — X min)/ (X max — X min)

Sometimes this is not enough, e.g.\ if having a few extreme values, and an approach is desired which better reflects the average data.
Here the standardization, also called z-score normalization, comes into play.
This method scales the values so that the mean value is placed at \tb{0} and the standard deviation is placed at \tb{1}.
%https://medium.com/analytics-vidhya/what-are-data-standardization-and-data-normalization-f880dd9e79b6

\subsubsection{Padding}
\label{subsubsec:padding}

Padding is a technique to adapt input sequences or matrices of different dimensions to the same size.
For example if one screen only has five \gls{ui} elements, and the next one has twenty, the input size varies significantly.
Therefore, there exists three approaches to overcome this problem~\cite{baeldung_padding}.
The first one is to extend all inputs of a specific feature to the longest available input dimension of this feature.
Then every sequence has to be filled with additional values, almost always \tb{0} is used for that, until it matches the longest dimension (\ti{same padding}).
Another technique called \ti{valid padding} cuts all data values after reaching the smallest dimension of all inputs of the feature.
One can also use a combination to e.g.\ pad all inputs with zero to the mean dimension, but throw away all exceeding values as they aren't expected to contribute to a better result.
\ti{Causal padding} is a form of \ti{same padding}, but it prepends the fill values in front of the sequence.
This is helpful if having \gls{rnn}s, which rely on a lengthier inputs in order to predict their next value, without need of filtering out short sequences.

\subsubsection{One-Hot-Encoding for Categorical Values}
\label{subsubsec:categorical_variables}

A categorical value must be treated differently than numerical ordinal values.
This is demonstrated best on a concrete example.
Imagine a dataset with pictures of animals, and we want to categorize their species.
Then the values of all possible species like \code{dog} and \code{cat} is called \ti{vocabulary}.
To write it in a table, one could add a column called \code{species} and write their species as a \ti{string} (cf.\ table~\ref{tab:raw_data_table}).
Unfortunately a \gls{ann} cannot work with \ti{strings}, but only with numerical values.
So one could think that mapping each species to a number can solve this issue.
Indeed, this is possible for ordinal categories, which have a strict linear order, such as ratings or gradings.
They are then treated the same as \tb{integer values}.
But animals aren't structured ordinal, thus each species must be treated equally.
To reflect that we split the species column in multiple sub-columns, so that each new one represents one species, see table~\ref{tab:one-hot}.
If the animal belongs to that species, one inserts a \tb{1} or \tb{True} and if not, a \tb{0} or \tb{False} is filled.
This is also called \gls{gl-one-hot}-encoding ~\cite{brownlee2021}.

\begin{table}[htbp!]
    \begin{subtable}[c]{0.5\textwidth}
        \centering
        \begin{tabular}{|l|l|l|}
            \hline
            \tb{$I_{d}$} & \textbf{Img} & \textbf{Species} \\
            \hline
            0 & \ti{blob} & cat \\
            1 & \ti{blob} & dog \\
            2 & \ti{blob} & cat \\
            3 & \ti{blob} & horse \\
            \hline
        \end{tabular}
        \subcaption{Raw data table}
        \label{tab:raw_data_table}
    \end{subtable}
    \begin{subtable}[c]{0.5\textwidth}
        \centering
        \begin{tabular}{|l|l|l|l|l|}
            \hline
            \tb{$I_{d}$} & \tb{Img} & \tb{Cat} & \tb{Dog} & \tb{Horse}\\
            \hline
            0 & \ti{blob} & 1 & 0 & 0 \\
            1 & \ti{blob} & 0 & 1 & 0 \\
            2 & \ti{blob} & 1 & 0 & 0 \\
            3 & \ti{blob} & 0 & 0 & 1 \\
            \hline
        \end{tabular}
        \subcaption{One-hot encoded species}
        \label{tab:one-hot}
    \end{subtable}
    \caption{\Gls{gl-one-hot}-encoding of categories, $I_{d}$ is the index of the data entry}
    \label{tab:cat_one_hot}
\end{table}

\subsection{Types of Machine Learning}
\label{subsec:machine-learning-types}

To get be able to evaluate which subfield of \gls{ml} is the best choice, we need an overview of the existing approaches listed in \cite{ghahramani2003unsupervised}.

\tb{Supervised Learning} is used for \ti{Classification} and \ti{Regression} tasks.
It uses \ti{labeled} samples, which means the input $x$ and output $y$ is known beforehand.
Essentially it learns through comparing the actual output (labels), which is provided, with the output the model \ti{predicts}.

Unlike in Supervised Learning, \tb{Unsupervised Learning} isn't provided any labels $y$.
It is used for clustering, dimensionality reduction or anomaly detection. %may need to explain more
E.g.\ in clustering the \gls{ml} model detects similar objects and groups them together.
What is seen as similar is up to the model, so it is not known before, with what groups or clusters the model comes up with.

The algorithms using \tb{Semi-supervised Learning} seek to adopt from unlabeled and labeled data~\cite{van2020survey}.
E.g.\ for a classification problem, the model is either trained first with unlabeled data resulting in clusters, which are then provided a class with a few labeled samples.
Or classes are trained based on labeled data, but their class boundaries are extended through unlabeled data, which follows the approach of clustering.
%It can be assumed that the model is trained first with labeled data, e.g.\ to be able to differentiate certain clusters.
%Then the model predicts the labels of the unlabeled data which is then serving as new data point to train the model.
%This helps to enlarge the cluster or move its boundaries.
%Test points to be classified by the model into the clusters then have a smaller distance to the previous predicted point,
%and therefore adapting the label of it's cluster, instead of being assigned to a different cluster.
%It can be seen as a model trained with classification through labeled data and then extended with clustering through unlabeled data.
%Also, if having only few labeled points, one can start with training a cluster through unlabeled data and then determine the class label of the clusters with the labeled data points.

%\tb{Self-supervised Learning}
\todo{may explain Self-supervised learning}

\tb{Reinforcement Learning} is closely related to decision theory \cite{ghahramani2003unsupervised}.
A model is given the input and produces an output, which is exposed to the environment.
The environment then provides a reward or a punishment to the next learning iteration of the model.
This then tries to maximise the rewards and minimize the punishments.
This learning type can also be helpful in fine-tuning \gls{rnn}s.

\section{Classes of Artificial Neural Nets}
\label{sec:neural-net-classes}

\todo{activation functions,Gradient decent, backpropagation and may include graphic of perceptron}
Unfortunately the concepts of neural networks, cannot be explained in detail, as it is beyond the scope of this thesis.
A good choice to make familiar with Neural nets is the book \quotes{Neural Networks and Deep Learning} of Nielsen~\cite{nielsen2015neural}, which is consulted throughout this section.

An \gls{ann} uses biological neuron systems as paradigm to generate mathematical models.
The biological neuron model was transferred to computer science and is called \ti{perceptron model}.

A perceptron -- or also named neuron in our context -- is fed by one or multiple inputs $x_i$ and performs a function $f$ on them resulting in an output $\hat{y}$, where $i$ is an index of the $n$ input values.
Also, a weight $w_i$ and a bias $b_i$ can be applied to each of these inputs.

\begin{quote}
    \begin{math}
        \hat{y} = \sum_{i=0}^{n} x_i w_i+b_i
    \end{math}\newline
\end{quote}

Multiple of these neurons can be used together producing one or multiple outputs, thus forming a neural layer, which can act as \gls{nn} together with the input and the output layer.
If multiple neural layers (also called \ti{hidden layers}) are connected to each other, it is called a \ti{deep neural net}.
The output(s) $\hat{y}$ of one or multiple neurons then serve as the input(s) $x$ of one or multiple neurons in the next layer.
The last layer -- the output layer -- then serves the values, which have to be predicted.
These values can be compared to the real data (labels) using an error function like \gls{rmse}.
The whole model can be seen as a cost function, which has to be minimized during training.
In the training process weights $w$ and biases $b$ for each neuron are adapted, in order to produce outputs, which matches best the labels.
This is done iteratively using an optimizer (Adaptive) \gls{gl-gradient-descent} and \ti{backpropagation}, by searching the minimum cost in multiple steps, called \ti{epochs}.

%- can solve tasks by abstraction or generalization of data relations
%
%
%Activation Functions
%Cost function
%Gradient
%- Regression: Continuous Values
%- Classification: Multiple class
%- One Class

%\subsection{Convolutional Neural Nets}

\subsection{Recurrent Neural Networks}
\label{subsec:rnn}

LSTMs / GRU -> supervised

\subsection{Dense Layer}
\label{subsec:dense-layer}

\subsection{Embedding Layer}
\label{subsec:embedding-layer}

% https://machinelearningmastery.com/use-word-embedding-layers-deep-learning-keras/
As mentioned in section~\ref{subsubsec:categorical_variables} categories can be represented using \gls{gl-one-hot} encoding.
The conversion can result in innumerable amount of columns, which is equivalent each having its own feature dimension.
To reduce the dimensionality of such encoding, so-called categorical \ti{embeddings} can be introduced~\cite{brownlee2021}.
This means that each single species is represented by a vector.
The length of the vector can be selected freely, it is called the \ti{embedding dimension}.
A lookup-table is created which encodes each category with randomly initialized weights of size of the embedding dimension (cf.\ table~\ref{tab:species-embedding}).
Now the embedding vector for each species is looked up and replaced in the data table resulting in an embedded representation (cf.\ table~\ref{tab:embedding_data_table}).
During training of the model the weights of the look-up table are successively updated (like for the dense layer ~\ref{subsec:dense-layer}) to reduce the error (loss) of the overall model.
An embedding would result in an \gls{gl-one-hot}-encoding, if the embedding dimension and the vocabulary size are equal and the conversion matrix (lookup-table) is the identity matrix.

\begin{table}[htbp!]
    \begin{subtable}[c]{0.5\textwidth}
        \centering
        \begin{tabular}{|l|l|l|l|}
            \hline
            \tb{$I_{d}$} & \tb{Img} & \tb{SP\_1} & \tb{SP\_2}\\
            \hline
            0 & \ti{blob} & 0.1 & 0.6 \\
            1 & \ti{blob} & 0.4 & 0.8 \\
            2 & \ti{blob} & 0.1 & 0.6 \\
            3 & \ti{blob} & 0.5 & 0.5 \\
            \hline
        \end{tabular}
        \subcaption{Species encoded with embedding}
        \label{tab:embedding_data_table}
    \end{subtable}
    \begin{subtable}[c]{0.5\textwidth}
        \centering
        \begin{tabular}{|l|l|l|l|}
            \hline
            \tb{Species} & \tb{$I_{s}$} & \tb{SP\_1} & \tb{SP\_2}\\
            \hline
            0 & cat & 0.1 & 0.6 \\
            1 & dog & 0.4 & 0.8 \\
            2 & horse & 0.5 & 0.5 \\
            \hline
        \end{tabular}
        \subcaption{Lookup table for the 2-dimensional embedding of species}
        \label{tab:species-embedding}
    \end{subtable}
    \caption{Embedding of categories, $I_{d}$ is the index of the data entry, $I_{s}$ is the index of the species}
    \label{tab:cat_embeddings}
\end{table}



According to \cite{alam2019impact} these steps can be removal of emoticons, elimination of stopwords and stemming for text based models.

Category Embedding before LSTM
% https://stackoverflow.com/questions/47217151/keras-lstm-with-embedding-layer-before-lstm-layer
% https://stackoverflow.com/questions/52627739/how-to-merge-numerical-and-embedding-sequential-models-to-treat-categories-in-rn/52629902#comment136040845_52629902
% https://stats.stackexchange.com/questions/270546/how-does-keras-embedding-layer-work
% https://stackoverflow.com/questions/47868265/what-is-the-difference-between-an-embedding-layer-and-a-dense-layer


Embedding dimension is about the actual voc\_size, but not too large.
Dimension near the actual average length of features (?)

\begin{comment}
\subsubsection{Transformer}
\label{subsubsec:transformer}
%https://en.wikipedia.org/wiki/Transformer_(machine_learning_model)
\todo{Check}
The neural net transformer model is a type of deep learning architecture that uses attention mechanisms to process sequential data, such as natural language or speech.
It does not rely on recurrent or convolutional layers, which are commonly used in other neural network models.
Instead, it uses a combination of self-attention, multi-head attention, and feed-forward layers to encode and decode the input and output sequences12

The difference between encoder and decoder transformer is that they have different roles and sublayers in the model.
The encoder transformer takes an input sequence, such as a sentence in one language, and converts it into a vector representation, called an encoding, that captures the meaning and context of the input.
The encoder transformer consists of multiple identical layers, each with two sublayers: a multi-head self-attention layer and a feed-forward layer.
The self-attention layer allows the encoder to learn the relationships and dependencies between the words in the input sequence.
The feed-forward layer applies a non-linear transformation to the output of the self-attention layer

The decoder transformer takes the encoding from the encoder and generates an output sequence, such as a sentence in another language. The decoder transformer also consists of multiple identical layers, each with three sublayers: a masked multi-head self-attention layer, a cross-attention layer, and a feed-forward layer. The masked self-attention layer allows the decoder to learn the relationships and dependencies between the words in the output sequence, but prevents it from attending to the future words that have not been generated yet. The cross-attention layer allows the decoder to attend to the encoding from the encoder and learn the alignment between the input and output sequences. The feed-forward layer applies a non-linear transformation to the output of the cross-attention layer
\end{comment}

\subsection{Autoencoders}
\label{subsec:autoencoder}
Encoder, Decoder

\section{Tensorflow and Keras}
\label{sec:tensorflow-keras}
Layers
FlattenLayer

Positive Integer to Dense Vectors of fixed size

\section{Evaluation and Metrics}
\label{sec:evaluation-metrics}

\begin{comment}
Mit einer Konfusionsmatrix (\textit{Confusion Matrix}) werden die Vorhersage\-ergebnisse zu einem Klassifizierungsproblem zusammengefasst.

\begin{table}[!htbp]
    \centering
    \begin{tabular}{ |c|c|c| }
        \hline
         & Prediction positive & Prediction negative \\
        \hline
        Actual positive  & \gls{tp} & \gls{fn} \\
        \hline
        Actual negative & \gls{fp} & \gls{tn}  \\
        \hline
    \end{tabular}
    \caption{Confusion Matrix.}
    \label{tab:confusion-matrix}
\end{table}

Precision $P$: How many samples were correctly predicted positive, from all positively predicted samples.
Also, the probability of correctness of the classification, if the sample is predicted as positive.
\begin{equation}
    P = \frac{\text{\gls{tp}}}{\text{\gls{tp} + \gls{fp}}}
\end{equation}

Recall or Hit Rate $R$: How many samples were correctly predicted positive, from all actually positive samples.
Also, the probability of correctly classified as positive, if the sample is actually positive.
\begin{equation}
    R = \frac{\text{\gls{tp}}}{\text{\gls{tp} + \gls{fn}}}
\end{equation}

For email spam recognition, precision is important, as if an email is classified as spam, it should be correctly predicted.
For disease recognition, recall is much more important, as if a patient isn't recognized as positive, the outcome is much more drastically.
\end{comment}
\begin{comment}
    Accuracy, Precision, Recall
    \subsection{\texorpdfstring{F\textsubscript{1}-Score}{F1-Score}}

    \label{subsec:f1-score}

    Die F\textsubscript{1}-Wertung ist eine der am häufigst verwendeten Metriken zur Evaluation von \acrshort{ml}-Modellen. Um den F\textsubscript{1}-Wert berechnen zu können, werden die Werte der im vorherigen Kapitel beschriebenen Confusion Matrix benötigt. Die Formel zur Berechnung des F\textsubscript{1}-Wertes lautet:

    \begin{equation}
        F\textsubscript{1} = 1 / \frac{\frac{1}{P} + \frac{1}{R}}{2}
    \end{equation}

    Dazu werden die Formeln für den Wert von \textit{Precision} und \textit{Recall} benötigt.

    Der Wert der Precision-Gleichung wird durch den Anteil an richtig vorhergesagten positiven Ergebnissen (\acrshort{tp}) auf die Gesamtheit aller als positiv vorhergesagten Ergebnissen berechnet.
    Mit der Recall-Gleichung, welche auch Hit-Rate genannt wird, wird mit der Anzahl
    \acrshort{tp}-Ergebnisse geteilt durch die Gesamtheit der tatsächlich positiven Ergebnisse,
    der Wert der Gleichung berechnet. Dieser Wert gibt den Anteil der Objekte der positiven Klasse an, der durch ein \acrshort{ml}-Modell errechnet wurde.
    Der F\textsubscript{1}-Wert hat einen Wert zwischen [0, 1]. Je höher dieser Wert ist, desto besser ist die Leistung des \acrshort{ml}-Modells~\cite{evaluationsMethoden}.
\end{comment}

Accuracy, precision and recall, as well as the combined F\textsubscript{1}-Score is only applicable for categorical labels, thus classification.

For regression tasks, what we are aiming for, the \gls{mae}, the \gls{mse} and the \gls{rmse} are suited much better.

\begin{quote}
\begin{math}
\text{RMSE}(y, \hat{y}) = \sqrt{\sum_{i=0}^{N - 1} (y_i - \hat{y}_i)^2 / N }
\end{math}\newline
with the number of samples $n$, the actual value $y$, and the predicted value $\hat{y}$ \cite{MAE_RMSE}.
\end{quote}

The \gls{mae} is the mean value when summing all absolute errors $y_i - \hat{y}_i$ and divide them by $n$
The \gls{mse} squares the single absolute errors in the \gls{mae} to punish single outliers.
The \gls{rmse} is the root of the \gls{mse} to avoid squaring units, but also provide a realistic error values over all samples.

%\subsection{Under and Overfitting}
%\label{subsec:under-and-overfitting}
As \ti{loss} is meant the cost or the error, which is differentiating the label from the prediction.
In most cases the loss function is the \gls{mse} which must be optimized, e.g. with \gls{gl-gradient-descent}.
%--
%\todo{explain Validation loss} % https://www.baeldung.com/cs/learning-curve-ml
%Loss = cost
%function
%---

As explained in \ref{sec:neural-net-classes} \gls{ann}s can be trained in epochs.

If a model was trained too few epochs, it probably is not able to grab the general trend or learn enough about the data to built context.
This is called \tb{Underfitting}.
The opposite \tb{Overfitting} is the case, if model was trained too many epochs.
This results in knowing the train dataset very well and being able to predict every sample in the training set.
But as soon as it is confronted with new data samples, it results in a larger errors than in earlier epochs.
This can be solved by adding an \ti{early stop} function, which automatically interrupts the training, when the validation loss \ref{todo} increases.



\chapter{Related Work}

Describe relevant scientific literature related to your work.

\section{UI Tree and Datasets}

\subsection{ERICA}

ERICA is a design and interaction mining application, which allows gathering \ti{interaction traces} by capturing the users activity on Android apps~\cite{deka2016erica}.
This is accomplished through a web-based interaction layer in contrast to the other common approach of using \ti{accessibility services} directly.
They justify that approach by the lack of need to install additional applications, as only a browser is required.
A further reason is the response latency of the commonly used \ti{UiAutomator}, which cannot collect the data in time.
Also they argue that capturing and simultaneously interacting with the apps may overload the user device and challenges the user experience.
Therefore the much more powerful servers take the task of capturing the UI trees.
The apps are hosted on multiple physical devices with a modified Android OS directly connected with the server.
ERICA captures UI screens and user flows by tracking UI changes.
They then used this data to form k-mean clusters from the UI elements (visual and textual features) and the interactive elements (icons and buttons).
Based on the clusters they then build classifiers and trained an AutoEncoder (\ref{subsubsec:autoencoder}) to determine the flows from the test dataset.
The authors worked out 23 common user flows (from over a thousand popular Android apps) which aim to provide complementary, promising or new design patterns and trends.

%- data-driven app design application
%- gathers user interaction trace > 1000 popular apps
%- 3000 flow examples

%\todo{descibe how they worked out the 23 user flows, which Autoencoder they used etc.}

\subsection{Rico / RicoSCA}

Rico \cite{deka2017rico} (spanish for \quotes{rich}) is the successor of ERICA.
It aims to help perform better at designing and support the creation of adaptive UIs.
As far as known to date this is the largest collection of mobile app designs and traces with covering 72k UI screens in 9.7k Android apps.
Like its predecessor Rico uses a web-based approach to collect user traces.
It enables the applications like searching for designs, generation of UI layouts and code, modeling of user interactions, and prediction of user perception.
It exposes visual, textual, structural, and interactive design properties of more than 72k unique UI screens.
Unfortunately the dataset doesn't include interaction traces for app to app transitions or interactions with the Android OS itself.
In table~\ref{tab:rico_view_hierarchy_attributes} a collection of all view hierarchy attributes is shown with their meaning.
These were extracted by iterating over all view hierarchy files contained in the traces of the dataset.
This gives insights in what attributes were recorded in the Rico dataset and what relevance they may have during training the model.
\todo{Explain AutoEncoder of Rico and difference to current approach}

The RicoSCA dataset has been formed out of the research topic of mapping language instructions to mobile UI action sequences~\cite{li2020mapping}.
They removed screens whose bounding boxes in the view hierarchies are inconsistent with the screenshots with the help of annotators.
The process of filtering resulted reduced the Rico dataset to 25k screens.


\begin{table}
  \small
  \centering
  \begin{tabular}{|l|c|c|>{\RaggedRight}p{0.5\linewidth}|}
    \hline
    \tb{Key} & \textbf{Type} & \textbf{Shape} & \textbf{Description} \\
    \hline
    \multicolumn{4}{c}{Per View} \\
    \hline
% Annotated by word2vec
%    \_is\_leaf\_node & bool & (1) & \\
%    \_caption\_preorder\_id & bool & (1) & \\
%    \_caption\_depth & bool & (1) & \\
%    \_caption\_node\_id & bool & (1) & \\
%    \_caption\_postorder\_id & bool & (1) & \\
    activity\_name & string & (1) & Name of the activity: e.g. \quotes{com.my\_app.AppName.MainActivity} \\
%    added\_fragments & [] & (None) & \\
%    active\_fragments & [] & (None) & \\
    is\_keyboard\_deployed & bool & (1) & Indicates if the keyboard is shown \\
    request\_id & int & (1) & Id used by the crawler to request the view \\
    \hline
    \multicolumn{4}{c}{Per Node} \\
    \hline
    abs-pos & bool & (1) & Indicates if position in \ti{bounds} is relative or absolute; if \ti{true}, \ti{rel-bounds} is set \\
    adapter-view & bool & (1) & Indicates that children are loaded via an adapter, see~\cite{android_adapterview} \\
    ancestors & [string] & (None) & Ancestors of current node, e.g. \quotes{android.view.View} \\
    bounds & [integer] & (4) & Absolute or relative boundaries, dependent on \ti{abs-pos} \\
    children & [node] & (None) & Child nodes \\
    class & string & (1) & \quotes{com.my\_app.lib.ui.views.DropDownSpinner} \\
    clickable & bool & (1) & User can interact by press / click \\
    content-desc & string & (1) & (Accessibility) description of the node \quotes{Interstitial close button} \\
    draw & bool & (1) & Indicates if this node is drawn on the canvas \\
    enabled & bool & (1) & Indicates if this node is in the enabled state \\
    focusable & bool & (1) & Indicates if this node can be focused \\
    focused & bool & (1) & Indicates if this node can is currently in focus \\
    font-family & string & (1) & States the font family, e.g. \quotes{sans-serif} \\
    long-clickable & bool & (1) & Indicates if this node has a long press action \\
    package & string & (1) & States which packages the node belongs to \quotes{com.my\_app.mypackage} \\
%    pointer & string & (1) & \todo{Presumably the saving address in the memory, e.g. \quotes{92690f4}} \\
    pressed & bool & (1) & Indicates if this node can is currently pressed \\
    rel-bounds & [integer] & (4) & Relative boundaries, if \ti{abs-pos} is set to \ti{true} \\
    resource-id & string & (1) & The unique resource identifier for this view \quotes{android:id/navigationBarBackground} \\
    scrollable-horizontal & bool & (1) & Indicates if this node can be scrolled horizontally \\
    scrollable-vertical & bool & (1) & Indicates if this node can be scrolled vertically \\
    selected & bool & (1) & Indicates if this node can is currently selected \\
    text & string & (1) & Text value if this node is a textual element \\
    text-hint & bool & (1) & Explanation text for text boxes or icons \\
    visibility & string & (1) & Indicates if this node is hidden, e.g. \quotes{visible}, \quotes{gone}\\
    visible-to-user & bool & (1) & Indicates if this node can be seen in the viewport by the user \\
    \hline
  \end{tabular}
  \caption[Attributes of a view hierarchy record]{Collection of attributes of a \ti{view hierarchy} record, extracted from all interaction traces of the Rico \cite{deka2017rico} dataset.}
  \label{tab:rico_view_hierarchy_attributes}
\end{table}

See \cite{deka2017rico}

\subsection{Mobile UI CLAY Dataset}

Learning to Denoise Raw Mobile UI Layouts for Improving Datasets at Scale

\begin{itemize}
  \item Provides a so-called \ti{CLAY} pipeline which denoises mobile UI layouts from incorrect nodes or adding semantics to it.
  \item better than heuristic approach
  \item results are dynamic and out of sync, invisible objects, misaligned, in the background (greyed out)
  \item aim: \quotes{large scale high quality layout dataset}
  \item 37.4 \% of the screens contain invalid objects
\end{itemize}

See \cite{clay}
https://github.com/google-research/google-research/tree/master/clay


\section{Vector models}

\subsection{Doc2Vec and Word2Vec}
%http://proceedings.mlr.press/v32/le14.pdf

\subsection{Screen2Vec}

\subsection{Screen2Words}

\subsection{Intention2Text}

\subsection{Html2Vec}

\subsection{Tree2Vec}

\subsection{Activity2Vec}

\section{Time Series / Sequence models}
\subsection{Seq2Seq Model}

\subsection{Click Sequence Prediction in Android Mobile Applications}
\cite{lee2018click}
-> No code or dataset

\subsection{Large-Scale Modeling of Mobile User Click Behaviors Using Deep Learning}
\cite{zhou2021large}
-> No code or dataset



\chapter{Methodology}

% \section{Apparatus}

% \section{Procedure}

% \section{Measurements}

% \section{Participants}

\section{Android UI Data}
\subsection{Data tree structure}


\includegraphics[width=\textwidth]{graphics/android_layout_inspector}
https://developer.android.com/studio/debug/layout-inspector

\subsection{Retrieval of UI data via Android Accessibility Service}

Semantics tree:
https://developer.android.com/jetpack/compose/semantics
https://android.googlesource.com/platform/frameworks/testing/+/jb-dev/uiautomator/library/src/com/android/uiautomator/core/AccessibilityNodeInfoDumper.java

\autoref{lst:ListingANDlstlisting,android_accessibility_node} zeigen, wie man Programmlistings einbindet.
https://github.com/mimuc/app-ins-gruene

\lstinputlisting[language=XML,label=android_accessibility_node,caption={Android Accessibility Node in XML.},float]{code/android_accessibility_node.xml}

\section{Machine Learning}
\subsection{Preprocessing}
\subsubsection{Normalization, Feature selection}

Such as Filtering privacy invasive details

Parameterizing the vectorization process
             a) Vector length
             b) Weighting of features
             c) Manipulating individual parameters of model
             
\subsection{Supervised vs Unsupervised vs Semisupervised}
\subsection{Under and Overfitting}
\subsection{Evaluation Metrics}
\subsection{Neuronal Nets}
Activation Functions
Cost function
Gradient
- Regression: Continous Values
- Classification: Multiple class
- One Class


Tensors
%https://stackoverflow.com/a/48599383/5164462

LSTM 4 dimensional
% https://stackoverflow.com/questions/54743549/is-it-possible-to-making-lstm-model-with-4-dimension-shape-of-data

Embedding before LSTM
% https://stackoverflow.com/questions/47217151/keras-lstm-with-embedding-layer-before-lstm-layer
% https://stackoverflow.com/questions/52627739/how-to-merge-numerical-and-embedding-sequential-models-to-treat-categories-in-rn/52629902#comment136040845_52629902

TimeDistributedLayer
% https://stackoverflow.com/a/61588937/5164462
% https://stackoverflow.com/questions/53107126/what-are-the-uses-of-timedistributed-wrapper-for-lstm-or-any-other-layers
\subsubsection{Deep Neuronal Nets}

\subsubsection{Convolutional Neuronal Nets}
\subsubsection{Recurrent Neuronal Networks and LSTMs / GRU}

\subsection{Layers}

- Embedding layer
    Dimension near the actual average length of features
- Dense Layer

Positive Integer to Dense Vectors of fixed size

\subsubsection{Autoencoders}


\chapter{Results}

Proof of concept, as an example gesture or click traces were used as labels.
More general approaches can be used such as a description of the intent \cite{screen2words}, which needs another data set.

%\todo{overall goal is more than just interaction traces}
Make the algorithm as independent of the data as possible.
Find general rules to feed the data.
Applicable to other research fields, not just UI traces.
Use LSTM, so predict something unseen, in contrast to RICO or ERICA, which only categorize the current context

%Steps:
%- Data acquisition
%- Data cleaning / Preprocessing (Panadas)
%- Split into Training Data, Validation data, and Test data (cannot adapt the model after using the test data)
%- Train the model with the train data
%- Evaluate the model with the test data, then can adapt the model by the developer
%- Last deploy the model to production

\section{Datasets}

- Problem with sequential data sets

\subsection{Rico}

\begin{itemize}
  \item Too less frames.
  \item No transition between apps.
\end{itemize}

\section{Preprocessing Android UI tree data}
\subsection{Filtering privacy invasive details}

- rico doesn't use logins or any privat data
- gestures can tell more about the user
-


\subsection{Normalization, Feature selection}

Dealing with variable length data tf.io.VarLenFeature()

\section{Model}

%- Screen2Words

LSTM 4 dimensional
% https://stackoverflow.com/questions/54743549/is-it-possible-to-making-lstm-model-with-4-dimension-shape-of-data

Limitations to only 3 dimensions, needs flattening

Sample dimension (X -> y)
Time (Step) Dimension
Feature Dimension
Data, Quantity dimension, such as Image dimensions, or multiple nodes

TimeDistributedLayer
% https://stackoverflow.com/a/61588937/5164462
% https://stackoverflow.com/questions/53107126/what-are-the-uses-of-timedistributed-wrapper-for-lstm-or-any-other-layers


Multiple approaches

\todo{create graphic for each approach}

AutoEncoder:

\begin{itemize}
  \item Encoder -> Decoder -> LSTM -> Decoder
  \item Encoder -> LSTM -> Decoder
  \item LSTM -> Encoder -> Decoder (AutoEncoder)
\end{itemize}

Decoder can either only decode to x and y or to whole UI tree.

\section{Evaluation}

\begin{figure}[htbp!]
  \centering
  \includegraphics[width=\textwidth]{graphics/model_history_loss}
  \caption{Model loss vs validation loss}
  \label{fig:model_history_loss}
\end{figure}

\subsection{Mean Squared Error}

\section{Limitations}

Dataset
Dataset is not through different apps, only in one app.
Dataset is not detailed enough in the time steps, or not containing all data
Dataset is not long enough
Dataset has no paid apps or apps with login, which most services require
Dataset has wrong data see \cite{clay}

Preprocessing
Need more time to validate what are the core parameters to predict the next user intent


Model needs more investigation on what data is needed
How many neurons are required to achieve this
Play around with different layers, also Convolutional and pretrained embeddings



\chapter{Application of Android UI Tree Vectors}

\section{Automation and Testing of Android Apps}
\section{UI Design Similarities}
\section{Action Prediction Models, User Behavior Modeling}
\section{Behavioral Analyses for Smartphone Usage Patterns}


\section{What is a suitable model for the prediction of user intent?}.
\todo{formulate as topic}

To answer this question, LSTM has to be compared against other methods of predicting user intents.
As shown in \todo{add ref} classic stochastic approaches may can predict larger scope of the user such as the next app or a general user workflow.
But they are not sufficient for predicting views, screens, or even precise gesture inputs.
To overcome this limitation the \gls{ml}-models have established in large datasets.
As the described problem is to be contextualized in the prediction of time series, the following options are offered:
\begin{itemize}
    \item Simple \gls{rnn}
    \item \gls{gru}
    \item \gls{lstm}
    \item \gls{gl-transformer}
\end{itemize}

Simple \gls{rnn} is limited in the capacity of establishing long-term semantics.
\gls{gru} is missing the forget gate compared to the \gls{lstm} making it simpler and faster, but may perform weaker on complex datasets.
The \gls{gl-transformer} model has many advantages, such as fast and efficient training, parallelism of input sequences and recognizing long-term patterns through multi-head attention.
On the other hand few prior work was done in the mobile sector which covers prediction of \gls{ui} trees.
Therefore, no publicly available coding approach was found which could be extended.
Also, as described in~\cite{zhou2021large} the structure of the model is quite complex -- with two transformers -- and needs more processing steps in general.
\gls{lstm} is well documented and the common choice to predict time dependent series.
It is well-supported by Keras~(\ref{keras}) and easy to use.
Also, similar approaches have been made in the area of app prediction or app summarization~\ref{cite me}, which can be used as basis for this work, such as Screen2Words~\ref{subsec:screen2words}.


\section{At what level of detail the predictions can be made?}
\todo{formulate as topic}

As noted in the introduction the term \ti{intent} has a very wide scope.
It's prediction can only be made in fractions, or serve as indicator.
The semantically closest way and also the most detailed would be to describe the users intent in words, thus a description of what the user wants to do next.
Unfortunately, the users intent description cannot be determined yet, as no according dataset is provided.
An important step, the screen summarization as shown in Screen2words~\ref{subsec:screen2words}, already was made, which could also be fed with the users intention descriptions.
But this would then only reflect the already passed fulfilled intent and not the upcoming next purpose.
Also, the users flows can be predicted as shown in the works of ERICA~\ref{subsec:erica}.
Next app prediction already works quite well~\ref{todo}, which also reflects the larger intent of the user, but is not very detailed.

In order to approximate that goal the assumption was made that the next user interaction also give hints on what the user is intending to achieve.
The interaction is resulting in another screen, which then the user (hopefully) intends to see.
So not only the interaction, but also the next contents, such as screen or single views can give a hint to the intent.
As it turns out, this is possible, if replacing the labels in the proposed model, but presumably it will not be very precise.

%category → app → screen → view → action
1. Predict Gestures, prediction of current screen, not next screen, is possible and can easily calculate distances between this and the next point
2. Predict Screen or parts of screens, labels not gestures, box positions, text prediction, decoded and compared to current output, more a qualitative evalutation,
3. Predict App -> Not possible with RICO, no cross app traces

\section{How the user can be supported with their tasks on digigal end devices?}
\todo{formulate as topic}


\chapter{Conclusion and Future Work}
\label{sec:conclusion}

\section*{Summary}

In this thesis the Android UI structure and the ways of retrieving them has been worked out, to be able to gather a meaningful dataset.
The basics of \gls{ml}, \gls{nn}s, the preprocessing steps and evaluation metrics have been illustrated to grasp the idea of the proof-of-concept.
%The research methods have been stated to ensure the
The word intent has much room for interpretation, thus a set of indicators were given to better differentiate.
A proof-of-concept for an intent prediction model based on a \gls{lstm} was worked out, which does not yet predict advantageous gestures, but only predicts slightly better mean click coordinates.
The environment did not enable the performance to fully test out the potential.
However, the proposed concept provides flexibility and extensibility for future work and is accessible for the public.
More promising approaches arose, like transformer multi-attention model, actionable element prediction, and the usage of metrics like relative rankings (\ref{subsec:user-click-behaviors-deep-learning-transformer}).
The available technologies already enable large parts of intent prediction.
Nonetheless, more precise or semantically meaningful prediction models, such as screen or descriptive ones are still missing.

\section*{Outlook}
%Future directions for research in this area

Although the results of the proposed model did not quite match the expectations, this field of application has lots of open questions to be researched on.
The prospects are by adding more semantics, such as language features or sensory inputs, any scope of intent can be predicted more granularly.
The data selection also has lots of potential to improve by applying preprocessing steps such it is done in RicoSCA \cite{li2020mapping} or Clay \cite{clay}.
%Use LSTM, so predict something unseen, in contrast to RICO or ERICA, which only categorize the current context
Also the types of elements (class) have a much more impact on the outcome than the gesture positions.
%the proof of concept did not show the expected results, thus it can be learned a lot
%\todo{TAKEAWAY: click sequence may NOT a good indicator for where the user clicks as buttons can differ from app to app, more semantics, better relative actionable button click rate}
A public accessible app tracing app and a big enough dataset which also cover cross-app traces would be a good basis for future research.
Other model types have to be evaluated or reproduced.
% TAKEAWAY: Create custom dataset
%Generate Dataset which overcomes the limitations
Also, a user study should be conducted to evaluate how helpful an overlay can be, which proposes the next item to click.
Today, smartphones are expected to be powerful enough and are equipped with intelligent processors to take tasks like recording the accessibility tree and user interaction prediction.
This would also help preserving privacy.
Zhou and Li \cite{zhou2021large} already provided a \quotes{Next Click Overlay} that reduces the number of traversals from 9.04 to 2.61.
Feedback and improvement on such a visualization of a prediction system would be very interesting.
Applying reinforcement learning (section \ref{subsec:machine-learning-types}) to the existing models also would help improving the user experience.
%group of users

%Work out user flows, like in ERICA, but without the need to separate it from the interaction tree
%Take in visual and textual context (semantics).

%Use dataset with
%Also use accessibility service, as phones are much more powerful.
%No need for web interface.
%-> Reinforced directly on the phone, more privacy.
%
%ChatGPT – Image Recognition – Limitation als Ausblick

In regard to the current development in large language models like Bart or ChatGPT, it will be exiting to see what surprising breakthrough comes up next.


\newpage
\todo{check if references are still on their own page}
\newpage
\printbibliography[heading=subbibliography]

%All links were last followed on March 17, 2018.

\appendix
% \input{latexhints/latexhints-english} % Can't compile with latexhints-english

\pagestyle{empty}
\renewcommand*{\chapterpagestyle}{empty}
\Affirmation
\end{document}
