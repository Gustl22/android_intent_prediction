\chapter{Conclusion and Future Work}
\label{sec:conclusion}

In this thesis the Android UI structure and the ways of retrieving them has been worked out, to be able to gather a meaningful dataset.
The basics of \gls{ml}, \gls{nn}s, the preprocessing steps and evaluation metrics have been illustrated to grasp the idea of the proof-of-concept.
%The research methods have been stated to ensure the
The word intent has much room for interpretation, thus a set of indicators were given to better differentiate.
A proof-of-concept for an intent prediction model based on a \gls{lstm} was worked out, which does not yet predict advantageous gestures, but only predicts slightly better mean click coordinates.
The environment did not enable the performance to fully test out the potential.
However, the proposed concept provides flexibility and extensibility for future work and is accessible for the public.
More promising approaches arose, like transformer multi-attention model, actionable element prediction, and the usage of metrics like relative rankings (\ref{subsec:user-click-behaviors-deep-learning-transformer}).
The available technologies already enable large parts of intent prediction.
Nonetheless, more precise or semantically meaningful prediction models, such as screen or descriptive ones are still missing.

\tb{Outlook}

Use LSTM, so predict something unseen, in contrast to RICO or ERICA, which only categorize the current context
more semantics and language features
types of elements (class) has much more impact than position.
intent prediction is possible but only
the proof of concept did not show the expected results, thus it can be learned a lot

Other model structures have to be evaluated

\todo{TAKEAWAY: click sequence may NOT a good indicator for where the user clicks as buttons can differ from app to app, more semantics, better relative actionable button click rate}

% TAKEAWAY: Create custom dataset

%\section*{Outlook}

Future directions for research in this area

ChatGPT – Image Recognition – Limitation als Ausblick

Generate Dataset which overcomes the limitations

Make a study with actual feedback on a prediction system, visualization
-> Learn faster and directly, Reinforced learning
group of users

Work out user flows, like in ERICA, but without the need to separate it from the interaction tree

Take in visual and textual context (semantics).

Make dataset with app overlapping traces.
Use dataset with preprocessing such as RicoSCA or Clay. Dataset has wrong data see \cite{clay}
Also use accessibility service, as phones are much more powerful.
No need for web interface.
-> Reinforced directly on the phone, more privacy.

Make a user study which can apply this model.
