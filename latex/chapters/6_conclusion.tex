\chapter{Conclusion and Future Work}
\label{sec:zusfas}

Use LSTM, so predict something unseen, in contrast to RICO or ERICA, which only categorize the current context
more semantics and language features
types of elements (class) has much more impact than position.
intent prediction is possible but only
the proof of concept did not show the expected results, thus it can be learned a lot

Other model structures have to be evaluated

\section*{Summary}
\section*{Outlook}

Future directions for research in this area

ChatGPT – Image Recognition – Limitation als Ausblick

Generate Dataset which overcomes the limitations

Make a study with actual feedback on a prediction system, visualization
-> Learn faster and directly, Reinforced learning

Work out user flows, like in ERICA, but without the need to separate it from the interaction tree

Take in visual and textual context (semantics).

Make dataset with app overlapping traces.
Use dataset with preprocessing such as RicoSCA or Clay. Dataset has wrong data see \cite{clay}
Also use accessibility service, as phones are much more powerful.
No need for web interface.
-> Reinforced directly on the phone, more privacy.

Make a user study which can apply this model.
