\chapter{Related Work}

Describe relevant scientific literature related to your work.

\section{UI Tree and Datasets}



\subsection{Mobile UI CLAY Dataset}

Learning to Denoise Raw Mobile UI Layouts for Improving Datasets at Scale

\begin{itemize}
  \item Provides a so-called \ti{CLAY} pipeline which denoises mobile UI layouts from incorrect nodes or adding semantics to it.
  \item better than heuristic approach
  \item results are dynamic and out of sync, invisible objects, misaligned, in the background (greyed out)
  \item aim: \quotes{large scale high quality layout dataset}
  \item 37.4 \% of the screens contain invalid objects
\end{itemize}

Test \cite{clay}

\section{Vector models}

\subsection{Doc2Vec and Word2Vec}
%http://proceedings.mlr.press/v32/le14.pdf

\subsection{Screen2Vec}

\subsection{Screen2Words}

\subsection{Intention2Text}

\subsection{Html2Vec}

\subsection{Tree2Vec}

\subsection{Activity2Vec}


\subsection{RICO / RicoSCA}

"Rico is a public UI corpus with 72K Android UI
screens mined from 9.7K Android apps. [...]
We manually removed screens whose view
hierarchies do not match their screenshots by ask-
ing annotators to visually verify whether the bound-
ing boxes of view hierarchy leaves match each UI
object on the corresponding screenshot image.This filtering results in 25K unique screens."

\section{Time Series / Sequence models}
\subsection{Seq2Seq Model}


