\section*{Abstract}
\label{sec:abstract}

\todo{THIS IS THE EXPOSÉ, no need to correct!!}
\todo{search keywords to replace: you, we, I, TAKEAWAY, ??}
\todo{check if all images and tables are referenced}
\todo{check if words from the glossary can be replaced in the text}
\todo{check if every section or chapter cross reference has a prefix "chapter" or "section"}


The intent of a user is a source of information, which is hard to accommodate.
In the context of using a smartphone, this can imply the next user action, the user flow, or the recommendation for the next app.
A flexible, open-source proof-of-concept for an intent prediction model based on a \gls{lstm} has been worked out.
It includes three models ClickOnly, SelectedFeatures and FeaturesClickShifted, which aim to predict the next user click.
All three are performing only slightly better than the statistically mean click gesture, trained on 15\% of the usable Rico dataset.
Criteria for a dataset suitable for Android intent prediction were determined.
Approaches, like transformer multi-attention model, actionable element prediction, and further metrics have potential to push the development of semantically meaningful prediction models.
