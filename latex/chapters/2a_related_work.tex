\chapter{Related Work}

Describe relevant scientific literature related to your work.

\section{UI Tree and Datasets}

\subsection{ERICA}

ERICA is a design and interaction mining application, which allows gathering \ti{interaction traces} by capturing the users activity on Android apps~\cite{deka2016erica}.
This is accomplished through a web-based interaction layer in contrast to the other common approach of using \ti{accessibility services} directly.
They justify that approach by the lack of need to install additional applications, as only a browser is required.
A further reason is the response latency of the commonly used \ti{UiAutomator}, which cannot collect the data in time.
Also they argue that capturing and simultaneously interacting with the apps may overload the user device and challenges the user experience.
Therefore the much more powerful servers take the task of capturing the UI trees.
The apps are hosted on multiple physical devices with a modified Android OS directly connected with the server.
ERICA captures UI screens and user flows by tracking UI changes.
They then used this data to form k-mean clusters from the UI elements (visual and textual features) and the interactive elements (icons and buttons).
Based on the clusters they then build classifiers and trained an~\ref{autoencoder} to determine the flows from the test dataset.
The authors worked out 23 common user flows (from over a thousand popular Android apps) which aim to provide complementary, promising or new design patterns and trends.

%- data-driven app design application
%- gathers user interaction trace > 1000 popular apps
%- 3000 flow examples

%\todo{descibe how they worked out the 23 user flows, which Autoencoder they used etc.}

\subsection{RICO / RicoSCA}

"Rico is a public UI corpus with 72K Android UI
screens mined from 9.7K Android apps. [...]
We manually removed screens whose view
hierarchies do not match their screenshots by ask-
ing annotators to visually verify whether the bound-
ing boxes of view hierarchy leaves match each UI
object on the corresponding screenshot image.This filtering results in 25K unique screens."



Use web interface to gain tree and interaction traces

\begin{table}
  \centering
  \begin{tabular}{|l|c|c|>{\RaggedRight}p{0.4\linewidth}|}
    \hline
    \tb{Key} & \textbf{Type} & \textbf{Shape} & \textbf{Description} \\
    \hline
    \multicolumn{4}{c}{Per Trace} \\
    \hline
% Annotated by word2vec
%    \_is\_leaf\_node & bool & (1) & \\
%    \_caption\_preorder\_id & bool & (1) & \\
%    \_caption\_depth & bool & (1) & \\
%    \_caption\_node\_id & bool & (1) & \\
%    \_caption\_postorder\_id & bool & (1) & \\
    activity\_name & string & (1, None) & Name of the activity: \quotes{com.my\_app.AppName.MainActivity} \\
%    added\_fragments & [] & (None) & \\
%    active\_fragments & [] & (None) & \\
    is\_keyboard\_deployed & bool & (1) & Indicates if the keyboard is shown \\
    request\_id & int & (1) & \todo{TODO} \\
    \hline
    \multicolumn{4}{c}{Per Node} \\
    \hline
    abs-pos & bool & (1) & Indicates if position in bounds is relative to the parent or absolute to the screen boundaries \\
    adapter-view & bool & (1) & Indicates that children are loaded via an adapter, see https://developer.android.com/reference/android/widget/AdapterView \\
    ancestors & bool & (1) & \todo{TODO} \\
    bounds & bool & (1) & \todo{TODO} \\
    children & [node] & (1) & \todo{TODO} \\
    class & bool & (1) & \todo{TODO} \\
    clickable & bool & (1) & User can interact by press / click \\
    content-desc & bool & (1) & \todo{TODO} \\
    draw & bool & (1) & \todo{TODO} \\
    enabled & bool & (1) & \todo{TODO} \\
    focusable & bool & (1) & \todo{TODO} \\
    focused & bool & (1) & \todo{TODO} \\
    font-family & bool & (1) & \todo{TODO} \\
    long-clickable & bool & (1) & \todo{TODO} \\
    package & bool & (1) & \todo{TODO} \\
    pointer & bool & (1) & \todo{TODO} \\
    pressed & bool & (1) & \todo{TODO} \\
    rel-bounds & bool & (1) & \todo{TODO} \\
    resource-id & bool & (1) & \todo{TODO} \\
    scrollable-horizontal & bool & (1) & \todo{TODO} \\
    scrollable-vertical & bool & (1) & Vertically scrollable \\
    selected & bool & (1) & \todo{TODO} \\
    text & bool & (1) & \todo{TODO} \\
    text-hint & bool & (1) & \todo{TODO} \\
    visibility & bool & (1) & \todo{TODO} \\
    visible-to-user & bool & (1) & \todo{TODO} \\
    \hline
  \end{tabular}
  \caption[Values of Rico Node]{Values of Rico Trace and Node -- see \cite{deka2017rico}}
  \label{tab:Ergebnisse}
\end{table}

See \cite{deka2017rico}

\subsection{Mobile UI CLAY Dataset}

Learning to Denoise Raw Mobile UI Layouts for Improving Datasets at Scale

\begin{itemize}
  \item Provides a so-called \ti{CLAY} pipeline which denoises mobile UI layouts from incorrect nodes or adding semantics to it.
  \item better than heuristic approach
  \item results are dynamic and out of sync, invisible objects, misaligned, in the background (greyed out)
  \item aim: \quotes{large scale high quality layout dataset}
  \item 37.4 \% of the screens contain invalid objects
\end{itemize}

See \cite{clay}
https://github.com/google-research/google-research/tree/master/clay


\section{Vector models}

\subsection{Doc2Vec and Word2Vec}
%http://proceedings.mlr.press/v32/le14.pdf

\subsection{Screen2Vec}

\subsection{Screen2Words}

\subsection{Intention2Text}

\subsection{Html2Vec}

\subsection{Tree2Vec}

\subsection{Activity2Vec}

\section{Time Series / Sequence models}
\subsection{Seq2Seq Model}


