\chapter{Methodology}

In the upcoming sections the methodology for this thesis is presented.
This includes the type of work,
the analysis process of related work,
how the data was selected,
what decisions were made to accomplish the research goals,
and how the research questions are answered.

% What research problem or question did you investigate?
Working with mobile \gls{os} and how the user can be supported by solving their tasks on digital end devices is the main booster to embrace to this topic.
This also raises the question, how much aid a user can be given, to what detail it is capable of and where are its limitations.
To this date and as far as known no open accessible prediction model was published to predict the user intent to the detail of their gesture inputs.
Also, it is not apparent which technologies and methods are suitable to develop such an intent prediction system.
In order to converge to these goals, the research questions were developed and the methodology for those is explained.

%Quantitative, qualitative, or mixed? Primary or secondary? Experimental or descriptive?
The thesis follows a mixed strategy to develop the research topic.
Qualitative statements were taken into account in order to explain existing approaches and describe theories.
But the experimental proof of concept was evaluated using quantitative metrics.
A study to evaluate the experimental findings was exceeding the scope of this master thesis, although it would be very interesting.

%Start with your overall approach to the research.
In order to have an overview of the subject, various research papers were worked through.
These then were classified by its relevance to the topic and if suitable, presented in related work (chapter~\ref{ch:related-work}).
Some of these papers included sample code and datasets, which were evaluated in parts, such as Rico (\ref{subsec:rico}), Screen2vec (\ref{subsec:screen2vec}) or Screen2words (\ref{subsec:screen2words}).
Also, tools were looked for, which can help accomplish the research.
To acquire the basics of \gls{ml}, a workshop has been passed, which teaches the handling with Tensorflow and Keras using the programming language Python.
The structure of the thesis was chosen by knowledge of working on previous thesis, the given template, and proposals from other scientists.

The goal is to develop an approach that can be followed by the reader, but also is detailed enough to understand the underlying technology.

%What type of data did you need to answer it?

%\section{Data Aquisition}
How you collected and analyzed your data
Describe the specific methods you used for data collection and analysis.
How did you collect and analyze your data?
What tools or materials did you use?
How did you ensure the quality and accuracy of your data?

- Why RICO, rather use a dataset with accessibility service.
- ERICA employs a human-powered approach over an automated one:
    - More realistic results
    - required user input, which cannot emulated, Google Captcha, real data
    - humans detect the completion of UI updates \cite{deka2016erica}
    - erica is quite outdated


Any tools or materials you used in the research.
The type of research you conducted

E.g. Google Scholar, Google Research, Tensorflow, Keras, Udemy, Open Source, Reproducible

\section{Methodological variety}
Explain why you chose these methods over others.
How do they relate to your research question and literature review?
How do they address the limitations or gaps in existing research?
How do they suit your research design and objectives?

- Keras vs PyTorch, Python vs other languages
- No similar approach
- No dataset present with consecutive sequential app usages
- Many different approaches to solve this problem:
    - Encoder, Decoder, etc...
- A Study can follow

\section{Methodological choices}
Evaluate and justify your methodological choices.
Why you chose these methods
How did they affect the outcome of your research?
What challenges or difficulties did you encounter and how did you overcome them?
How can you ensure the credibility and generalizability of your findings?

IntelliJ and Overleaf
\todo{overall goal is more than just interaction traces}
Goal: Make the algorithm as independent of the data as possible.
Find general rules to feed the data.
Applicable to other research fields, not just UI traces.
Use LSTM, so predict something unseen, in contrast to RICO or ERICA, which only categorize the current context


\section{Research Biases}
How you mitigated or avoided research biases

How this thesis is working?
Apparatus, Procedure, Utilities


\section{How research requestions are answered}
\quotes{What is a suitable model for the prediction of user intent?}.
\quotes{At what level of detail the predictions can be made?}
\todo{may add a third research question}
